\chapter{Conclusion}

In conclusion, in this dissertation, we presented a system that we developed to detect and recognition the traffic light and sign with the sensor hints of the smartphones.
This system can be useful for the pedestrian navigation, especially visually impaired.
We combine the sensor fusion with the image processing to efficiently detect traffic light in a subframe.
Firstly, we predict and track the traffic lights in a video frame using the orientation of the smartphone.
The subframe selection depends on the prediction of traffic light position using the sensor hints from the smartphone.
The subframe processing provides us computation time improvement and better accuracy.
We get in an order of magnitude improvement of computation time on average and our misdetection rate is significantly reduced.
Finally, we recognize the walk and stop signs for the pedestrian navigation that gives our system more robustness.
We achieve a significant accuracy for sign recognition using a neural network architecture.
In our system, although we use the model-based computer vision technique to detect the traffic light using the color and shape information.
Learning-based technique can also be combined with our algorithm with the sensor hints which can provide improved result with the improvement of the computation time.
%Nowadays many learning-based techniques are used to detect the traffic light that provides a  significant result.
%Finally, overall results may be improved if our traffic light detection algorithms with the sensor hints are combined with the learning-based algorithm for detecting traffic light states.


