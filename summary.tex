\summary

With the exponential growth of smartphone usage and its computational capability, there is an opportunity today to build a usable navigation system for the visually impaired. 
A smartphone contains virtually all sensors for sensing the surrounding environment such as GPS, cameras, and inertial sensors. 
However, there are many challenges for building a complete navigation system, such as low-level methods of environment sensing, accuracy, and efficient data processing.
In this dissertation, we address some of these challenges and present a system for traffic light detection, which is fundamental for pedestrian navigation by the visually impaired in outdoors. 
In this system, we analyze the video feed from a smartphone's camera using model-based computer vision techniques to detect traffic lights. 
Specifically, we utilize both color and shape information as they are the most prominent features of the traffic lights.
Additionally, we use the inertial sensors of a smartphone to compute the 3D orientation of a smartphone to predict a subpart of a video frame, which is highly probable to contain the traffic lights. 
By processing only that subpart, we improve the computational time by an order of magnitude on average. 
Furthermore, due to the processing of a subpart instead of the whole video frame, our system achieves higher accuracy because of reduced false positive.
We evaluated this system in various lighting conditions such as cloudy, sunny, and at night, and achieved over 95\% accuracy in traffic light detection.  


