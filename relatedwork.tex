\chapter{Related Work}
\label{c:relw}

Blind and visually impaired people can not have information about their location or the direction with respect to the traffic or obstacles on the way.
The conventional ways of guide dog and long cane can help to find out the obstacles on the way, not help to provide information of their position.
Navigation system helps people to travel in a convenience and independence.

S Ram at el. designed a system, People Sensor \cite{peoplesensor}, which uses pyroelectric and ultrasound sensor to distinguish between a person and object obstruction in the path of the user.
It helps to reduce the embarrassment through unintended contact with people and object in directional path.
There are many ways to find the location of the users.
Direct sensing 

%In this work, we build a navigation system to detect the traffic light in a public street for visually impaired.
%There has some existing systems for helping the blind and visually impaired find their way at indoor and outdoor.
After the introduction of Global Positioning System (GPS) at 1980s, many navigation assistant systems integrate GPS for visually imapaired.
Loomis was one of the first to propose the navigation system using DGPS and FM correction data reciever to get the location of the visually impaired \cite{loomis1,loomis,loomis2}.

There has some commertially released system for outdoor navigation for blind and visually impaired users.
Ariadne GPS \cite{arigps} developed by Ciaffoni is one of the first GPS apps for blind and visually impaired.
Other commertially released apps for iPhone and Android devices are BlindSquare \cite{blindsq} developed by MIPsoft and ViaOpta Nav \cite{viaopta} developed by Novartis Corporation etc.
These apps use GPS to inform users the current location, announcement of user points of interest and use open source map to navigate.
Seeing Assistant move app \cite{seeing} developed by Transition Technologies is the only GPS app for blind people that lets the user operate the app through speech commands.

Other systems that exploit GPS to find the user’s location are MoBic \cite{mobic}, BrailleNote GPS and Trekker developed by Humanware group \cite{human}, etc. BrailleNote GPS is commercially available and provides the user with nearby location names and the distance to destination along the path. 

However,the shortcomings of GPS are well kmown.
A GPS sensor is ineffective at indoor.
Furthermore, the GPS signal can not be tracked when blind people moves through tall buildings or high walls or trees.
Some studies proposed and implemented differential GPS which can provide better accuracy \cite{drishti2,gps}.
It is costly and need fixed ground station, only efficient for outdoors.



\cite{traffic_turan}
