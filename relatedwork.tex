\chapter{Related Work}
\label{c:relw}

Blind and visually impaired people can not have information about their location or the direction with respect to the traffic or obstacles on the way.
The conventional ways of guide dog and long cane can help to find out the obstacles on the way, not help to provide information of their position.
Navigation system helps people to travel in a convenience and independence.

S Ram at el. designed a system, People Sensor \cite{peoplesensor}, which uses pyroelectric and ultrasound sensor to distinguish between a person and object obstruction in the path of the user.
It helps to reduce the embarrassment through unintended contact with people and object in directional path.
There are many ways to find the location of the users \cite{survey}.

%In this work, we build a navigation system to detect the traffic light in a public street for visually impaired.
%There has some existing systems for helping the blind and visually impaired find their way at indoor and outdoor.
After the introduction of Global Positioning System (GPS) at 1980s, many navigation assistant systems integrate GPS for visually imapaired.
Loomis was one of the first to propose the navigation system using DGPS and FM correction data reciever to get the location of the visually impaired \cite{loomis1,loomis,loomis2}.

There has some commertially released system for outdoor navigation for blind and visually impaired users.
Ariadne GPS \cite{arigps} developed by Ciaffoni is one of the first GPS apps for blind and visually impaired.
Other commertially released apps for iPhone and Android devices are BlindSquare \cite{blindsq} developed by MIPsoft and ViaOpta Nav \cite{viaopta} developed by Novartis Corporation etc.
These apps use GPS to inform users the current location, announcement of user points of interest and use open source map to navigate.
Seeing Assistant move app \cite{seeing} developed by Transition Technologies is the only GPS app for blind people that lets the user operate the app through speech commands.

Other systems that exploit GPS to find the user’s location are MoBic \cite{mobic}, BrailleNote GPS and Trekker developed by Humanware group \cite{human}, etc. BrailleNote GPS is commercially available and provides the user with nearby location names and the distance to destination along the path. 

However,the shortcomings of GPS are well kmown.
A GPS sensor is ineffective at indoor.
Furthermore, the GPS signal can not be tracked when blind people moves through tall buildings or high walls or trees.
Some studies proposed and implemented differential GPS which can provide better accuracy \cite{drishti2,gps}.
It is costly and need fixed ground station, only efficient for outdoors.

Alternative approches have been proposed to assist the blind people, such as ultrasound \cite{drishti} or radio frequency identification (RFID) \cite{rfid} transponders or virtual blind cane to detect obstacles using laser and inertial measurement unit (IMU) \cite {virtual}

Although WLAN transmitters or RFID tags or WiFi is low cost technologies, installing them dedicatedly in the whole city would be expensive and inconvenient.
Slight changes of the scenes also hamper to the navigation process.

In order to be extinseively applicable, navigation system design needs to be wearable, low cost and mobile technology based.
To achieve this aim we propose a comuter vision based navigation system for blind and visually impaired.

The research on vision based localization system is active in recent years.
The map based navigation method require a global map to make a decision for the navigation \cite{online,map,map2}.
For this purpose, sequential images of the environment are registered in a database.
Then to get accurate location and orientation for real world images image -to-image matches with the database.
Another navigation method create a map while moving and then use that for navigation \cite{fly,fly2,fly3}. 
\cite{visual} proposed a system with wearable stero camera to estimate the egomotion using visual odometry method.



Traffic light detection is a important part of the outdoor navigation navigation for the visually impaired.
There is a significant research of traffic light detection for autonomus selfdriving car and driving assistence system \cite{traffic_turan,selfdrive,traffic,traffic2,traffic3}.
This system introduces a technique to detect traffic light state using vehicle localization and prior knowledge of traffic light location.


Traffic light color is an important feature to distinct it from the scenario.
To detect and recognize the color there is different work space.
The RGB color space is used \cite{rgb2}.
Because of the lightening changes problem, the color values of RGB change in different condition.
The other research group is working with the color space which is more immune to lightening condition.\todo{cite}
Although the HSV space is most used \todo{cite}, LUV space is also used \cite{luv}

For our system to detect color of traffic light we use the HSV space due to the description of color in HSV space is similar to the human perspective.

For outdoor navigation system traffic light detection is only a part of the system either for autonomus vehicle or pedestrian navigation.
It is important to use less time to detect the traffic light.
We can use the information from the sensor like GPS, accelerometer, gyroscope to get the position of the traffic light. \cite{sensor,sensor2,sensor3}
For the navigation purpose we need a portable system.
Now a days smartphone uses is growing and it has internal sensor that we can use to get the position of the traffic light.

For autonomus selfdriving car or driving assistance system, the position of traffic light is stable in respect to driving.\cite{signalguru}
This paper introduces system, Signalguru, which get the position of traffic light from the sensor data of smartphone.
Since the traffic light is always in the upper part of the scenario, they processed the upper half of the frame to detect the traffic light.
In context of pedestrian navigation, the camera of smartphone is not always fixed because of the movement of the body part while walking \cite{sensor_pedestrian,sensor_pedestrian2}.



