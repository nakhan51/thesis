\chapter{Related Work}
\label{c:relw}

In this section, we describe a brief introduction to the most notable research on the navigation for visually impaired.
Visually impaired people can not have information about their location or the direction with respect to the obstacles on the way.
The conventional ways of the guide dog and long cane can help to find out the obstacles on the way, but do not provide their position information.
Navigation systems help people to travel independently.

There has been a significant research \cite{survey} in localization and obstacle avoidance.
For obstacle avoidance, People Sensor \cite{peoplesensor} used pyroelectric and ultrasound sensor to detect an object the user's path.
%It helps to reduce the embarrassment through unintended contact with people and object in the directional path.
\todo{cite more about obstacle}
or virtual blind cane to detect obstacles using laser and inertial measurement unit (IMU) \cite {virtual}

%In this work, we build a navigation system to detect the traffic light in a public street for visually impaired.
%There has some existing systems for helping the blind and visually impaired find their way at indoor and outdoor.
After the introduction of Global Positioning System (GPS) at the 1980s, many systems integrated GPS for the navigation by the visually impaired.
Loomis at el. \cite{loomis1,loomis,loomis2} was one of the first to propose a navigation system using DGPS and correction data over FM to obtain accurate localization.

There are some commercial systems for outdoor navigation for blind and visually impaired users.
Ariadne GPS \cite{arigps} developed by Ciaffoni is one of the first GPS apps for navigation by the blind and visually impaired.
Some other commercially released apps for iPhone and Android devices are BlindSquare \cite{blindsq} and ViaOpta Nav \cite{viaopta}.
These apps use the GPS to inform users the current location, give an announcement of user points of interest and use open source map to navigate.
Seeing Assistant Move \cite{seeing} is the first app for blind people that lets the user to operate the app through speech commands.
Other systems that use GPS to find the user’s location are MoBic \cite{mobic}, BrailleNote GPS and Trekker \cite{human}. 
BrailleNote GPS is commercially available and provides the user with nearby location names and the distance to the destination along the path. 

The GPS provides good location estimations. 
However, there are several shortcomings of the GPS.
The GPS sensor is ineffective at indoors.
The GPS location error can also be high at urban canyons due to multi-path effect and obstructed view of the sky.
%% Some studies proposed and implemented differential GPS which can provide better accuracy \cite{drishti2,gps}.
%% It is costly and needs fixed ground station, only efficient for outdoors.
%% Furthermore, the GPS signal can not be tracked when blind people move through tall buildings or high walls or trees.

Since GPS does not work in indoors, there has been other approaches for localization in indoors using various instrumentation techniques. 
Researchers instrumented indoor areas with ultrasound \cite{drishti} or radio frequency identification (RFID) \cite{rfid} transponders to provide localization with triangulation.
Recently, due to the pervasive deployment of Wi-Fi networks, localization through Wi-Fi triangulation also became popular.  
Most of the localization techniques in indoors using instrumentation depends on fingerprinting where the error can be high.
Furthermore, small changes of the environment can reduce the accuracy.

In order to be extensively applicable, navigation systems needs to be highly accurate.
Additionally, the system needs to be wearable and low cost.
%To achieve this aim we propose a computer vision based navigation system for visually impaired.
To achieve this goal, there has been some recent work on vision based localization with smartphones or other wearable cameras and sensors.
\todo{cite vision based systems w/o map first}
The map based navigation methods \cite{online,map,map2} require a global map to make a decision for the navigation. 
In map-based localization systems, sequential images of the environment are registered in a database.
Then to obtain the location and orientation in the same area at a later time images are matched in the database.
In \cite{fly,fly2,fly3}, Simultaneous Localization and Mapping is used to create a map while moving along with the localization.
In \cite{visual}, the authors proposed a system with a wearable stereo camera for higher accuracy localization utilizing the depth information.

In outdoor streets, traffic light detection is an important part of the navigation system for the visually impaired.
There has been significant research on traffic light detection for autonomous driving and driving assistance systems \cite{traffic_turan,selfdrive,traffic,traffic2,traffic3}.
In \cite{traffic_turan}, the authors combined previously mapped traffic light locations along with the vehicle location to achieve reliable estimation of traffic light status.
In recent years, in addition to the model-based methods \cite{model,model2}, learning-based methods \cite{survey_traffic} became popular for traffic light detection.
The model-based approaches creates a heuristic model that rely on color or shape information.
These approves were dominant in the past decade.
The color information is significant for traffic light detection as color can be distinguished from the surroundings.
%Primarily to find the region of interest (ROI) and to classify the traffic light state we use the color information.
To detect and recognize the color, the detector defines the heuristic threshold of color in selected color space.
The RGB color space is most common as the input video frames are in this space \cite{rgb2}.
Because of the lightening changes problem, and the color and intensity information is mixed in all the channels of  RGB color space as we discussed in \S\ref{s:color_space}, the values of RGB channel change in different condition.
The other researcher on traffic light recognition is working with the color space which is more immune to lightening condition and hue distribution is much narrower \cite{hsv2}.
The color can vary in different conditions, so to make the model more robust, distinctive shape of traffic light is also used by applying the Hough transform on an edge map \cite{hough,hough2,signalguru} or by using radial symmetry \cite{radial,radial2}.
The shape information is fused with the color information, applying them as a filter after color segmentation \cite{signalguru}.
For our system to detect traffic light color, we use the HSV space due to the description of color in HSV space is similar to the human perspective and we use Hough circle transform to get the shape information of the traffic light. 

For outdoor navigation system traffic light detection is only a part of the system either for autonomous vehicle or pedestrian navigation.
It is important to use less time to detect the traffic light.
We can use the information from the sensor like GPS, accelerometer, gyroscope to get the position of the traffic light. \cite{sensor,sensor2,sensor3}
For the navigation purpose, we need a portable system.
Nowadays smartphone uses is growing and it has the internal sensor that we can use to get the position of the traffic light.

For autonomous self-driving car or driving assistance system, the position of the traffic light is stable with respect to the vehicle while driving.
\cite{signalguru} introduces a system, Signalguru, which get the position of the traffic light from the sensor data of smartphone.
Since the traffic light is always in the upper part of the scenario, they processed the upper half of the frame to detect the traffic light.
In the context of pedestrian navigation, the camera of a smartphone is not always fixed because of the movement of the body part while walking \cite{sensor_pedestrian,sensor_pedestrian2}.
So it is important to track the traffic light position from the sensor hints.
In our system, we use the sensor hints at each video frame to get the relative position of traffic light from the previous video frame and finally processed that area to detect the traffic light state.

Learning based model \cite{learning,learning2} is another category to detect traffic light states.
SVM classifier together with HOG features \cite{selfdrive} is one of the popular learning based detector in computer vision research areas.
The learning based aggregated channel feature (ACF) \cite{acf,acf2,lisa_cvpr} detector largely used in traffic light detection and provide outperform result.
Traffic light detection using deep learning is introduced in \cite{cnn,cnn2,cnn3}, where a convolutional neural network (CNN) model detects and recognizes traffic light states using the region of interest information from the smartphone GPS sensor.

In our system, we use the model based computer vision technique to detect and recognize the traffic light states.
Our main approach is to use the sensor hints to improve the computation time and the misdetection rate.
If we adopt these learning based approaches as our detection method with the sensor hint the result can be approved more.  





